\begin{abstract}
互联网不断发展,其中的信息也随着时间日渐增多,传统的返回检索方式开始无法满足获取所需信息和知识资源的全面性和以高效率完成。实体的知识关系抽取,可以从自然语言(中文文本)中抽取实体,并将实体之间的关系结构化,提高了用户可获取信息的全面性和获取的效率。

信息提取(IE)系统寻求从自然语言中提取语义关系文本,但大多数系统使用监督学习关系特定的例子,因此受到训练数据可用性的限制。开放式信息提取系统例如 TextRunner,在另一方面,致力于处理没有限制数量的从互联网获取的实体关系。

传统上,信息提取专注于精确、狭义的、预先指定的要求。例如从一些会议通告里提取时间和地点。而转移到另一个领域里,则需要用户对实体关系命名并手工制定新的提取规则或对新的训练集例子进行手工标注。这样的人力工作量随着目标实体关系的数量线性增加。

开放式关系抽取(Open Relation Extraction,ORE)是实体关系抽取的一种,它克服了传统信息提取(IE)的缺陷,即传统的信息获取技术对每种关系模式各自训练了他们的提取器。

有很多系统流行于英文的 ORE,例如 OLLIE,ReVerb 和 Exemplar 等。然而,对于其他语言的 ORE 则基本没有相关研究的报告。本毕业设计采用了基于语法分析的系统 ZORE(Zh ORE)来对简体中文文本进行关系和语义模式的抽取。ZORE 从自动解析的依赖树里定义了候选的关系,然后将实体的关系和语义模式不断地通过一种新的双重传播算法。

本文内容包括了对于所采取的实体关系抽取系统(ZORE)的介绍及其实现,以及关于 ZORE 所需组件的介绍,并将其应用在实际工程中。

\keywords{开放式关系抽取\zhspace{} ZORE\zhspace{} 双重传播算法}
\end{abstract}

\begin{enabstract}
With the continuous development of the Internet, with the passage of time, more and more information, the traditional return search method began to meet the need to obtain the required information and knowledge resources needs, fully and effectively completed. The knowledge of the entity can be extracted from the natural language (Chinese text), the structure of the relationship between the entities, and improve the user's available information is comprehensive and efficient.

The information extraction (IE) system seeks to extract semantic relations text from natural language, but most systems use specific examples of supervised learning relationships, thus limiting the availability of training data. On the other hand, an open information extraction system such as TextRunner is dedicated to handling unrestricted physical relationships obtained from the Internet.

Traditionally, information extraction focuses on precise, narrow and pre-defined requirements. Such as extracting time and place from certain meeting notifications. And move to another domain, the user needs to name the entity relationship and manually create a new extraction rule or manually annotate the new training set example. The human workload increases linearly with the number of target entities.

Open relational extraction (ORE) is an entity relationship extraction that overcomes the shortcomings of traditional information extraction (IE), the traditional information acquisition techniques for each relational model to develop their extractors.

English ORE has many popular systems, such as OLLIE, ReVerb and Exemplar. However, ORE in other languages ​​is basically no relevant research report. The graduation design uses a system based on parsing. ZORE (Zh ORE) simplifies the relationship between Chinese text and semantic models. Zore defines the candidate relationship from the automatic resolution dependency tree, and then passes the entity's relationship and semantic pattern continually through the new dual-propagation algorithm.

This article introduces the introduction and implementation of the Entity Relationship Extraction System (ZORE), and introduces the components required by ZORE and applies it to the actual project.

\enkeywords{Open relation extratction, ZORE, Double propagation algorithm}
\end{enabstract}

