% \chapter{总结与展望}
% 本文调研了当前对于中文的开放式关系抽取的研究现状,开放式关系抽取系统通过识别任意句子中的关系短语和相关参数,从文本中提取关系元组,而不需要预先指定的词汇表。然而,现代最先进的开放式关系抽取系统如 REVERB 和 WOE 有着两个共同的缺点,它们仅提取由动词介导的关系,并且它们忽略上下文,从而提取出了不正确的关系元组。首先,ZORE 通过提取由名词,形容词等介导的关系来实现较高的效率。 第二,上下文分析步骤通过在提取中包括句子中的上下文信息来提高精度,通过识别由名词和形容词介导的关系来扩展开放式关系抽取系统的句法范围。

% 最近已经提出了大量的开放关系提取方法,涵盖了从“浅”(例如,词性标签 POS)到“深”(例如,语义角色标签-SRL)的广泛的NLP机制。 一个重要的问题是NLP深度(和相关的计算成本)与有效性之间的权衡。

% 我们的实验发现,对于某些关系,开放式关系抽取的精确率可能不足。通过分析提取的上下文环境,开放式关系抽取系统能够识别许多和上下文有关的关系,但是是假设的或有条件的。 开放式关系抽取系统通过减少这些提取的置信度或通过归因和分类修饰符的形式将提取中的附加上下文相关联来提高精度。