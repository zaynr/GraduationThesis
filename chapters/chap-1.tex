\chapter{绪论}
\label{chap:introduction}

本章的内容包括对当前快速进步、扩张、提升下的互联网里对于实体关系的知识抽取调研的背景以及意义,阐释了在当先信息大爆炸的互联网时代里,通过从不计其数的信息中提取出实体,之后以实体之间关系的知识抽取对建立全面而准确的信息知识数据库,从而更好地服务互联网的使用者。

也包括了介绍国内外对于本课题的研究现状和本文的研究内容。

\section{研究背景与意义}
\subsection{研究背景}

传统的信息提取涉及很多人为因素的干扰,包括手工制定的规则或手工标注出来的样例来作为机器学习的输入数据,从而识别并判断在文本中两个实体之间的特定关系\citep{wang}。即使机器学习可以帮助枚举出潜在的可供提取的关系模式,但这个方式常常受制于提取已经被预定义的关系集。并且,这个方式不能适用于如今充斥着大量信息的互联网,特别是非结构化、未预先定义实体关系的信息。

另一种开放的信息抽取\citep{banko}方式独立于关系领域,从文本抽取关系并能方便地度量网络语料库的多样性和覆盖范围。语料库是开放式信息抽取系统所需要的唯一一种输入数据,它的输出是被抽取出的各种关系的集合。一个开放式信息抽取系统通过它的语料库大小来控制它所能抽取关系的可扩展性,并用以下方法来增加可处理的关系类型:基于浅分析、基于语法或没有预定义关系的基于词法分析的模式匹配\citep{wu2010, naka2012, etz2011}。现在的开放式关系抽取技术主要关注字面上关系的抽取,而没有尝试进行词法分析,但词法分析却是传统信息抽取的优点。

\subsection{研究意义}

